\section{Conclusion}
\label{sec:conclusion}
This project investigated the research question of how accurately a CNN can classify $\num{46}$ different European songbird species based on spectrograms computed from audio recordings. 
After building a robust and efficient data pipeline, different model architectures were tested, and hyperparameter optimization for the most promising model architecture was performed. 
For the evaluation, a total of five networks were trained using a stratified $k$-fold. The predictions on a previously separated test dataset showed an (balanced) accuracy score of
$\qty{90.02}{\percent}$ ($\qty{88.85}{\percent}$). To compare this approach to a simpler method, an ensemble of $k$-Nearest Neighbors classifiers 
was trained using $\num{45}$ audio features extracted from the raw waveforms of the audio. Here, the evaluation returned values of $\qty{33.49}{\percent}$ ($\qty{31.95}{\percent}$) for 
the (balanced) accuracy score. 
The convolutional neural network clearly outperforms the $k$NN classifier at this task. This is not surprising, as the $k$NN relies on pre-computed audio features 
that do not contain as much information as the $\num{162} \times \num{128}$ sized spectrograms used in the CNN, which is designed to learn the most meaningful features on its own.
Still, the $k$NN's performance is surprisingly good, compared to the statistical baseline of $\qty{3.34}{\percent}$. An advantage of the $k$NN classifier are the lower fitting and 
prediction times compared to the CNN. After the audio features are computed, the entire training process for the $k$NN takes just a few seconds, whereas the convolutional neural 
network requires several hours of training.\\ 
Considering the number of classes, the CNN demonstrated excellent performance in distinguishing between the different bird species. An increase in accuracy could potentially be 
achieved by using a more complex model architecture and further optimizing the hyperparameter space. However, this would require more computational resources and time. Additionally, 
a larger dataset and higher data quality could improve the results. Nevertheless, the research question can be answered positively: Modern machine learning algorithms, 
such as neural networks, are well-suited for complex classification tasks and can aid humans in research, such as monitoring bird species populations. In conclusion, 
the project can be viewed as a successful demonstration of the strengths and practical benefits of machine learning methods in real-world applications.

% - write summary
% - knn ist dogshit im vergleich zum CNN, aber actually garnicht so arsch
% - komplexe Problemstellung kann nicht gut durch einfache features gelöst werden -> braucht CNN
% - CNN performance ok, considering the number of classes
% - better perfromance with more data, deeper hyperparameter optimization, more complex models
% - would require a lot more computational resources and time
% - conluding: good results, strength of CNN is shown
