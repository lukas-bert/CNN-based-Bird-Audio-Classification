\section{Introduction}
\label{sec:introduction}
Even if it may not seem so at first glance, birds play a crucial role in our ecosystem. They connect different habitats by spreading seeds and transporting fish spawn. 
Just like bees and butterflies, some bird species, such as \textit{hummingbirds} and \textit{honeycreepers}, also significantly contribute to the pollination of fruits and flowers. 
Moreover, birds are an important part of our ecosystem's food chain: for example, most birds feed on invertebrates and thus act as nature's pest control units. \\
By monitoring bird populations, we can gain valuable insights into the health of our ecosystem.
A challenge that arises in this task is the difficulty of spotting birds, especially in dense vegetation.
However, their calls can often be heard over long distances without a direct line of sight.
Identifying birds by their calls is also challenging due to the vast number of species and subspecies, even in less biodiverse regions, especially among songbirds, 
which are hard to differentiate.
Moreover, classifying different bird species by sound requires expert knowledge and is a time-consuming process. \\
A solution to this problem could be provided by the automated classification of bird songs in soundscape recordings. Raw audio wave formats are difficult to handle due to their high data 
density and lack of interpretability. Hence, the common approach in analyzing audio files is to transition to the time-frequency domain and use 
spectrograms instead. 
A spectrogram is a visual representation of the time evolution of the frequency spectrum of a signal. It is created by performing a series of short-time Fourier transforms (STFTs)
\cite{Cohen1964} on a signal. The $x$-axis represents time, the $y$-axis represents frequency, and the intensity or color of each point represents the amplitude or power of the 
frequency at that time. To better reflect human perception of sound, the frequency spectrum can be transformed to the mel scale \cite{Stevens1937}, and amplitudes can be 
converted to decibels (logarithmic scaling). 
This is also beneficial in machine learning, where log mel spectrograms are used as input features for models such as convolutional neural networks (CNNs). 
Their ability to capture the characteristics of two-dimensional data makes them especially useful for tasks involving audio classification and recognition using spectrograms 
\cite{Chollet2017}.\\
This leads to the research question of this report:
\begin{center}
    \textbf{How accurately can different European bird species be classified by analyzing audio recordings using convolutional neural networks and audio processing?}
\end{center}
To investigate this question, this report presents the selection of a dataset in \Cref{sec:dataset} and a documentation of the solution approach in \Cref{sec:strategy}.
Five individual CNNs are trained using randomly picked spectrogram slices equivalent to $\qty{15}{\second}$ audio clips and used in an ensemble to predict audio recordings of 
different durations. The results of this study are evaluated and presented in \Cref{sec:results} and compared to an alternative method, making use of a simpler machine learning 
algorithm in \Cref{sec:alternative}. Finally, the discussion and conclusion are provided in \Cref{sec:conclusion}.


%- introduce importance for birds in the ecosystem
%    - connect habitats
%    - pest control / eat insects
%    - samen verteilen
%    - fisch leich verteilen
%    - maybe cite some study that shows importance of birds
%- challenges to monitor bird population etc.
%- required expert knowledge, time consuming
%- possibility to automatically detect species in soundscape recordings
%- use CNNs to analyze spectrograms because CNNs are very good at 2D data analysis
